
\section{引言}

\subsection{课题背景}
近年来,自动驾驶(或无人驾驶)成为一个研究热点。汽车自动驾驶主流技术体系是
高级汽车辅助驾驶系统(ADAS, Advanced Driver Asssistance Systems)。
ADAS 在多传感器信息融合技术的基础上,利用摄像头、雷达等传感器对车内外环境数据进行动态与静态的
识别、侦测、追踪,为驾驶者提供碰撞警告、高级停车辅助、车道保持辅助、离开车道警告、
换道辅助和驾驶员状态监视等功能 \cite{adas}。

由于车载嵌入式平台的性能有限,将实验室开发的车载软件应用直接移植到车载嵌入式平台上时,
不可避免地会遇到性能下降的问题,甚至会导致软件的某些功能无法正常使用。
因此,对于车载软件产品的开发来说,如何优化需要移植的软件产品的代码,并在保证其性能的情况下进行代码移植,
是一个值得研究的内容。

\subsection{研究问题的提出}
\subsubsection{项目介绍}
库位检测系统 \cite{huang2017} 作为自动泊车系统的应用之一,用于检测车身周围存在的停车位,并输出相应的停车位位置信息。

库位检测系统的主要功能为:将四路摄像头输入的影像拼接成一幅环视(鸟瞰)图;扫描环视图,从中识别出车库并标记。
检测出来的结果(即库位位置信息)被发送到其他应用程序用于进一步的规划控制,实现车辆的自主泊车。

目前该应用的算法已经在成熟的平台上开发完成,并要移植到目标机器(即车载嵌入式平台)运行。
相应项目所使用的嵌入式平台为Renesas的R-Car H2 \cite{rcarh2, rcarh2wiki} (以下简称RCar)。
此平台提供了如下软硬件环境:

{
\setstretch{1.0}
\begin{itemize}
    \addtolength{\itemindent}{2.5em}
    \item ARM Cortex-A15 四核中央处理器 \cite{arma15}
    \item CPU 带有硬件实现的浮点运算功能 \cite{armhf}
    \item PowerVR Series 6 G6400 图形处理器 \cite{pvrs6gpu}
    \item 带有 OpenMP 支持的专用 GCC 构建套件 \cite{openmp}
    \item OpenGL ES 3.0 及对应的着色语言支持 \cite{opengles3}
    \item 四部高清摄像头
\end{itemize}
}

库位检测系统已经在基于Windows的MATLAB \cite{matlab} 软件上开发并测试完成,而运行的目标机器是RCar这个嵌入式车载系统。
要在车载系统上运行这个库位检测系统,需要对代码进行移植。
为了便于代码修改与快速测试,这份代码还同时移植到了桌面版的Linux上。

\subsubsection{库位检测系统的初次性能测试}
代码在移植前已经由MATLAB转换为C++语言的代码,且成功在三个平台上编译通过并执行成功,因此这次测试使用对比测试。

三个平台的软硬件配置如下:

{
\setstretch{1.0}
\begin{itemize}
    \addtolength{\itemindent}{2.5em}
    \item Windows:Intel-i3 双核,独立显卡。
    \item Ubuntu:AMD-A10 四核,集成显卡。
    \item RCar:ARM-A15 四核,独立显示模块。
\end{itemize}
}

为减少干扰因素,库位检测统一使用相同的输入源。输出统一打印到日志。

在各平台运行后,分析运行日志,得到了表 \ref{tab:bench} 的数据。

\begin{table}[!hbt]
\centering
\caption{测试结果(单位:FPS)} \label{tab:bench}
\renewcommand{\arraystretch}{1.6}
\begin{tabular}{llrrr}
\hline 平台 & 编码 & 已识别 & 未识别 & 平均 \\ \hline
Windows & h.264 &  9.25926 & 3.55872 & 3.98664 \\
Windows & MJPG &  9.63054 & 3.40726 & 4.10280 \\
Ubuntu  & h.264 & 15.08250 & 7.59042 & 8.50587 \\
Ubuntu  & MJPG & 15.17239 & 7.90612 & 8.42602 \\
RCar    & h.264 &  4.20801 & 2.52262 & 3.02434 \\
RCar    & MJPG & --- & --- & --- \\
RCar    & *fact* & --- & --- & 24.0000? \\
RCar    & *svon* & --- & --- & 8.20374 \\
\hline
\end{tabular}
\end{table}

其中,fact编码表示的含义是Factory,即RCar平台配套自带的一个环视拼接程序。
svon编码表示的含义是Surround View Only,即RCar平台的软件只进行环视拼接,禁用库位检测模块时的性能。
h.264编码表示的含义是使用x264库重新压制的视频 \cite{h264}。MJPG则是AVI格式的一种内部编码 \cite{mjpg}。

\subsubsection{性能测试结果与问题的提出}
由上一节所展示的运行结果可见,初步移植的程序,与RCar自带的程序依然相差很大。

在库位检测这个应用场景下,实时性的要求较高。帧率平均在3.0fps的处理效率无法满足软件功能的要求,因此不能满足产品的设计需求。
另外,即使禁用了库位检测模块,只留下环视拼接的部分,运行效率依然远不如RCar提供的程序。

由RCar自带的环视拼接程序的表现可以知道,RCar平台确实能够提供充分的计算能力,并满足该应用的需求。
可以推断,当前的库位检测系统没能充分利用RCar的资源。因此需要分析两个程序的资源利用率。

使用RCar官方提供的工具进行系统监视,分别再次运行两个程序后发现,
RCar自带的程序同时使用了CPU与GPU,而当前的库位检测系统不仅没有使用GPU,CPU也只使用了一个核心。

要改善程序,首先想到的就是充分用好CPU的这四个核心。让CPU充分忙起来,可以大幅提高程序的效率。
此外还可以把目光转向GPU。GPU拥有非常多的计算核心,如果有办法将它们利用起来,进行简易却大批量的计算,性能必将有质的优化。

根据上述分析可知,为了提高移植后代码的性能,需要解决的主要问题有:

{
\setstretch{1.0}
\begin{itemize}
    \addtolength{\itemindent}{2.5em}
    \item 将原有的程序重构为多线程程序。
    \item 分析程序中使用的算法,进行特定的优化。
    \item 把存在高度并行化处理的算法的代码,重构为GPU代码。
\end{itemize}
}

重构为多线程程序,或者将一部分代码重构为GPU代码,需要切实了解代码中数据之前的相互依赖,避免出现数据冲突现象。
为此,在重构代码之前,首先要分析代码中各个数据的流向,找出它们之间的关系,为接下来的重构打好基础。

而为了简化数据流的分析,则要为程序的数据流生成合适的图。并在此图的基础上进行优化分析。

\subsection{本文所做的工作}
本文将提出一种初步的方案,用来帮助用户实现程序的数据依赖分析。本文主要做了以下工作:

{
\setstretch{1.0}
\begin{itemize}
    \addtolength{\itemindent}{2.5em}
    \item 找到合适的工具,用来分析C/C++语言编写的代码。
    \item 解析代码,分析数据流与控制流。
    \item 将代码的数据流与控制流转换为随机Petri网,以展现数据间的关系。
\end{itemize}
}
