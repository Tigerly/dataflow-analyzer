
\begin{center}
    \LARGE{\vspace*{1em}}
    \sffamily{基于LLVM与随机Petri网的程序模型生成工具研究}
    \LARGE{\vspace{1em}}

    \sffamily{\Large{摘要}}
\end{center}

随着科学技术的发展,社会上对智能硬件的需求越发高涨。人们希望智能硬件提供更加丰富多彩的功能。
现在图像识别、深度学习等技术广受欢迎,人们自然也会联想到它们运行在智能硬件上的图景。

然而,智能硬件基本属于嵌入式平台。嵌入式平台的单个处理器(CPU)效率一般不会太高,要运行图像识别这样的算法会很吃力。
不过,新的高端嵌入式平台使用多核心处理器,甚至会使用运算能力更加强大的图形处理器(GPU)。
利用好这些计算资源,就能显著改善代码性能,让代码能完成更多、更复杂的任务。
将原有代码进行并行化重构,可挖掘更多的计算资源。

并行程序重点关注各变量的数据依赖。以数据依赖为指引,可简化代码的重构过程。
人工分析数据依赖,耗时耗力。若能使用机器辅助完成这项工作,就可以减少人工工作量,提高重构效率。

本文所述内容,即利用LLVM这个成熟平台,基于车载嵌入式平台上的库位检测项目,完成数据依赖图的生成。

\vspace{1em}

\textbf{关键词:}\textnormal{LLVM,随机Petri网,数据依赖,数据流,控制流,并行化,性能优化}
