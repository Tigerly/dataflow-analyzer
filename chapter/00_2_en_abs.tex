
\begin{center}
    \LARGE{\vspace*{0.3em}}
    \LARGE{Program structure model graph generation \\ based on LLVM and stochastic PetriNet}
    \LARGE{\vspace{1em}}

    \Large{ABSTRACT}
\end{center}

The demand for smart hardware is increasing, with the growth of technology.
People hope smart hardwares can do more complex jobs.
Nowadays, image recognition, in-depth learning etc. are popular.
So they are naturally wanted to run on smart hardwares.

However, smart hardwares are basically embedded platform.
Performance of one CPU is not good on embedded systems.
It is hard to run an algorithm of image recognition.
Fortunately, last high-level embedded systems use multi-core CPU,
even have powerful GPU supporting general computing.
Making full use of these computing resources, can significantly improve code performance.
This will let the software do more complex jobs.
Parallelizing legacy code is a way to make the computing resources fully used.

Parallel code focuses on data dependency.
With the guide of data dependency, we can simplify the effort of refactor.
Using the machine instead of manual to analyze data dependency,
can save much time, improve the efficiency of refactor.

This article introduces a method.
We use LLVM, develop a plugin for it.
Basing on a image recognition project to generate a graph of data dependency.

\vspace{1em}

\textbf{Keywords: }\textnormal{LLVM, Stochastic PetriNet, Data Dependency, Dataflow, DFG, Parallelize, Optimize}
